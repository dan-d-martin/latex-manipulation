%% LyX 2.2.2 created this file.  For more info, see http://www.lyx.org/.
%% Do not edit unless you really know what you are doing.
\documentclass{article}
\renewcommand{\familydefault}{\rmdefault}
\usepackage[T1]{fontenc}
\usepackage[utf8]{inputenc}
\usepackage{geometry}
\geometry{verbose,tmargin=2cm,bmargin=2cm,lmargin=2cm,rmargin=2cm}
\usepackage{color}
\definecolor{note_fontcolor}{rgb}{0.800781, 0.800781, 0.800781}
\definecolor{shadecolor}{rgb}{0.777344, 0.777344, 0.777344}
\usepackage{array}
\usepackage{rotfloat}
\usepackage{calc}
\usepackage{framed}
\usepackage{multirow}
\usepackage{amsmath}
\usepackage{graphicx}
\usepackage{setspace}
\doublespacing

\makeatletter

%%%%%%%%%%%%%%%%%%%%%%%%%%%%%% LyX specific LaTeX commands.
%% Because html converters don't know tabularnewline
\providecommand{\tabularnewline}{\\}
%% The greyedout annotation environment
\newenvironment{lyxgreyedout}
  {\textcolor{note_fontcolor}\bgroup\ignorespaces}
  {\ignorespacesafterend\egroup}

%%%%%%%%%%%%%%%%%%%%%%%%%%%%%% User specified LaTeX commands.
 \usepackage{pdfpages}
\usepackage[english]{babel}
\usepackage[strict]{changepage}
\strictpagecheck
\usepackage{adjustbox}
\usepackage{rotating}
\usepackage{colortbl}
\usepackage{booktabs}
\usepackage{subfig}

\usepackage[style=apa,natbib=true,backend=biber]{biblatex}
\DeclareLanguageMapping{english}{english-apa}

\AtEveryBibitem{%
  \clearfield{issn} % Remove issn
  \clearfield{doi} % Remove doi

  \ifentrytype{online}{}{% Remove url except for @online
    \clearfield{url}
  }
}

\newlength{\totalleftmargin}
\newcommand{\calctotalleftmargin}{%
  \setlength{\totalleftmargin}{\dimexpr+\hoffset+1in+\leftskip}%
  \checkoddpage%
  \addtolength{\totalleftmargin}{\ifoddpage\oddsidemargin\else\evensidemargin\fi}%
}

\definecolor{lightgray}{gray}{0.7}
\definecolor{verylightgray}{gray}{0.9}
\addbibresource{/home/david/Phd/Thesis/ThesisRefs.bib}

\makeatother

\begin{document}
\selectlanguage{english}

\part*{Comprehension and Retention of Temporal Information from Graphical
and Textual formats : A Laboratory Study }

\emph{This document is a very early draft - the Also the citation
processor does not seem to handle web site references - this needs
investigation }

\pagebreak{}

\section{Rationale }

Time is an intangible concept, in a frequently quoted paper \citet{mctaggart_unreality_1908}
argues, from a philosophical point of view, that it is unreal. In
day to day, pragmatic terms however, time is clearly a part of all
of our lives and many metaphors are used to express and consistently
communicate it and its effects. \citet{lakoff_metaphors_2003}, considered
time in the context of their Conceptual Metaphor Theory (considered
by \citet{gibbs_evaluating_2011} (p530) to be the ``dominant theoretical
framework'' in the field) and as well as examples of financial metaphors
(e.g. ``Time is Money'') (p 7), provide examples relating time and
space; both those that consider time being the 'Moving Object' (e.g.
\textquotedbl{}Time flies\textquotedbl{}, p 42), and those that consider
the subject moving through time (e.g. ``As we go through the years'',
p 43). %
\begin{lyxgreyedout}
(\emph{Lots more to incorporate here but very conflicting ideas to
balance - maybe better to have detailed discussion in literature section}
of thesis?)%
\end{lyxgreyedout}

When communicating chronological information, such as historical events
and interactions, this can be presented in many ways, for example,
in narrative text, in tables or lists, or in diagramatic forms. This
research seeks to identify any significant comprehension and learning
differences between chronological information presented in text form
and presented graphically in the form of timelines.

Presenting temporal information in timeline form has a long history
itself. \citet{rosenberg_cartographies_2013} provide many examples,
in a variety of forms, ranging from a Greek chronological table “The
Parian Marble” dating back to 264/3 BCE (p.14) through to modern day,
web based, approaches (e.g. p.240) and discuss the notion of a 'map
of time' to help readers understand history. Little empirical research
has taken place however concerning the efficacy of timelines in aiding
the understanding and learning of history despite strong promotion
of their use in schools {[}They should be employed within every lesson''
(\citealp{hodkinson_primary_2011}, p. 6), “Timelines are basic tools
for developing knowledge and understandings about chronology'' (\citealp{pickford_timelines_2011},
p. 25), ``much can be achieved now in improving pupils’ sense of
time by teachers regularly using timelines'' (\citealp{maddison_developing_2011},
p. 9){]}. Indeed \citet{thornton_effects_1988} observe that “Little
can be confidently claimed about the effects of time lines on children's
understandings'' (p. 79) while \citet{burny_towards_2009-1}, consider
the associated instructional practices in time-related competences
to be ``driven by ideology, faddism, politics and marketing'' rather
than by empirical evidence (p. 484).

Some studies have examined specific aspects of the use of timelines
in a scientific manner: \citet{prangsma_multimodal_2007} studied
the use of multi-modal representations (visual and textual) in collaborative
history learning and analysed learning outcomes by means of testing,
and the underlying collaborative interactions via dialogue analysis.
\citet{korallo_middle_2010,korallo_use_2010} examined the use of
virtual reality timelines, exploring the engagement of spatial memory
in learning, and Masterman and Rogers \citeyearpar{masterman_framework_2002},
examined the use of a computer based multimedia timeline to teach
historical chronology in a primary school context. To date however
there has been no research identified that directly examines the comprehension
and retention differences between text and graphical versions of temporal
information.

When examining where differences may arise due to the two information
formats, or the cases where these may be complementary, it is necessary
to consider associated theories of cognition. Amongst these the most
directly relevant is the ``Cognitive theory of Multimedia Learning
CTML'' developed by \citet{mayer_cognitive_2005}, and supported
in his many subsequent studies \citep{mayer_determining_2009,massa_testing_2006,deleeuw_comparison_2008}.

The CTML combines three major elements:-
\begin{itemize}
\item Paivio's Dual Coding Theory (DCT) \nobreakdash- \citep{paivio_dual_1991,clark_dual_1991}
\item Sweller's Cognitive Load Theory (CLT) \nobreakdash- \citep{sweller_cognitive_1988,sweller_element_2010}
\item Previous work by \citet{mayer_learning_1996} and \citet{wittrock_generative_1992}
concerning active learning and organising/integrating new information
with existing knowledge
\end{itemize}
Based upon the notion of separate (but linked) channels for processing
verbal and non verbal/symbolic information \citep{paivio_dual_1991,clark_dual_1991},
each having limits \citep{sweller_cognitive_1988,sweller_element_2010},
and with connections being subsequently made in long term memory \citet{mayer_learning_1996}
and \citet{wittrock_generative_1992} , the CTML, seeks to maximise
the effective load and hence level of understanding by optimally using,
but not exceeding, the capacity of each channel. \emph{}%
\begin{lyxgreyedout}
\emph{(need something about positive effects of difficulties though?)}%
\end{lyxgreyedout}
\emph{.} Although the term 'multimedia' in the CTML is often studied
in relation to moving media/animation \citet{aldalalah_modality_2012,tabbers_multimedia_2004}
or to the combination of text/pictures and audio \citet{van_den_broek_effects_2014,leahy_when_2003},
a diagram with textual labels/elements is also considered in the theory
to be a form of multimedia, since it utilises both the visual and
verbal channels of DCT \citet{mayer_systematic_1989,bui_enhancing_2015,davenport_when_2008,ortegren_examining_2014,morrison_exploring_2015}.
This study compares the comprehension and learning effects resulting
from similar information being presented in linear text or in multimedia
in the form of timeline diagrams with textual elements/labels.

When considering comprehension and learning from text are either XXXXXXXXXXXXXXX
SOMETHING ON TEXT COMP - leading to diagram/map comp XXXXXXXXXXXXXXXXXXXXX

xxxxx) When considering a timeline as a 'map of time' however the
intangibility of time again needs to be taken into account. Geographical
maps have the benefit of being closely related to tangible physical
forms and earlier point regarding Geographical maps come in many forms
Cognitive linkage between time and space have been studied in a variety
of ways; \citet{miles_mapping_2010,bender_mapping_2014,bonato_when_2012,casasanto_space_2010,casasanto_time_2008,weger_time_2008}

\noindent\begin{minipage}[t]{1\columnwidth}%
\begin{shaded}%
Hypothesis 1: Visual (graphical) timelines result in significantly
different learning outcomes (retention and understanding) when compared
to textual versions. \end{shaded}%
\end{minipage}

\bigskip{}

\noindent\begin{minipage}[t]{1\columnwidth}%
\begin{shaded}%
Hypothesis 1A: Spatial ability and graphical preferences will positively
correlate with higher learning outcomes (both comprehension and retention)
 for graphical presentations \end{shaded}%
\end{minipage}

\bigskip{}

The aim of this research is therefore to identify any significant
comprehension and learning differences between chronological information
presented in text form and presented graphically in the form of timelines.
Both initial comprehension (understanding) and level of retention
will be measured to assess differences between the two forms of presentation.
In addition self assessments of cognitive load will be compared with
measurements of response time, and individual differences (including
spatial ability assessed through spatial rotation rates) will be examined
for correlation with primary results.

\section{Method}

\subsection{Design}

The experiment was a split plot design \citep{dean_design_1999}with
graphical/textual presentation being the independent variable, with
two levels, and two dependent variables (accuracy and response time).
Measurements were made in during two sessions, a week apart. The first
session involved two distinct conditions since measurements were taken
during during presentation of the graphical and textual material,
to assess comprehension, and shortly after the material had been removed
(to assess immediate retention levels). The second session purely
assessed retention and the original material was not made available.
Fictional materials were used to remove prior knowledge as a direct
confound with two different stories produced in both textual and graphical
form. Counterbalancing was used in the design such that during the
comprehension part of the first session the participant was randomly
presented with one of the two stories, in one of the two modes (graphical/textual),
and then with the other story in the other mode, while, during the
retention measurements, the selection of which story to ask questions
about first was also randomised. The use of laboratory conditions
minimised distraction etc. as another potential confound. Figure \ref{fig:Experiment-Overview}
provides a pictorial overview showing the two sessions. The primary
data relating to Hypothesis 1 results from asking similar sets of
questions on two different stories with one being presented in textual
form the other in graphical. These primary tests are labeled as 'Comprehension',
'Immediate Retention', and 'Retention' in Figure \ref{fig:Experiment-Overview}
and each of these has two associated sets of data (the marks and the
reaction times (RT)). To produce data for hypothesis 2 measurements
of spatial ability and visual/verbal preferences etc. were collected
across both sessions. (During pilot trials the complete set of spatial
tests was considered to take too long by many participants and so
was split across the two sessions). In the first session the participant's
demographic information was collected, their self rated vividness
of visual imagery \citep{marks_visual_1973,andrade_assessing_2013},
and Visualiser/Verbaliser and Multimedia Learning Preferences \citep{massa_testing_2006}
were collected and an assessment made of their spatial ability using
a set of rotated shapes as used by \citep{shepard_mental_1971} and
rotated letter (F) as used by \citet{cooper_time_1973}. During the
second session a second set of rotation tests was also performed,
together with two tests of Object Location Memory \citet{eals_hunter-gatherer_1994}.
The spatial tests also acted as distractor tasks during the first
session to clear any short term memory of the fictional stories, between
the comprehension element and the initial retention questions, something
considered necessary by \citet{karpicke_expanding_2007}.

\begin{figure}
\includegraphics[scale=0.51]{Figures/Experiment-outline_v0_3}\caption{\label{fig:Experiment-Overview}Experiment Overview}
\end{figure}


\subsection{Participants}

The study participants were recruited via the Department of Psychology
Research Participation Scheme (RPS) at a large university in the southwest
of the UK, where first year students in the Psychology department
undertake a total of 5 hours of research participation for credits.
A total of 52 participants completed the experiment, the age range
was 18 to 43 years with a mean age of 20.13 years (SD = 4.39) and
the majority (85\%) falling into the 18-20 year range. The gender
split was 41 (79\%) female. Forty three of the participants had used
timelines before and 40 (77\%) made frequent use of maps/diagrams/graphs
(at least once a month) in their job/studies/hobbies. The highest
education level for the majority (42, 81\%) was A-Level (with 2 holding
diplomas, 6 having degrees, and 2 doctorates).

\subsection{Materials}

The experiment was performed via a computer based questionnaire using
the open source Limesurvey software (Version 2.05+ Build 141210) which
was modified to provide response time measurements on all relevant
question types (the standard software only provides this for simple
response types such as ``yes/no'', and ``list selection'').
The graphical timelines were produced from XML representations of
the text using the Simile widget from MIT as described by Butler,
Gilbert, Seaborne, and Smathers \citeyearpar{butler_data_2004} with
the resulting images captured by screen capture for incorporation
into the questions. Although it was possible to produce the questionnaire
pages one at a time, the earlier pilot studies showed that, to cope
with the number of pages involved ( 308 in first session and 172 in
the second) and also ensure consistency, a scripted approach was needed.
The pages were therefore produced by a set of scripts, using standard
texts/images, and combined in counterbalanced groups. The selection
of the route through the questions (4 routes in the first session,
2 in the second) made use of the inbuilt random number capability
of the Limesurvey software.

The texts and graphical timelines can be seen below and were balanced
as closely as possible in terms of reading comprehension scores Flesch-Kincaid
Grade (Kincaid, Fishburne, Rogers, \& Chissom, \citeyear{kincaid_derivation_1975}),
Flesch Reading Ease \citep{flesch_new_1948}, number of dates, names,
and causal linkages etc.

\noindent\fbox{\begin{minipage}[t]{1\columnwidth - 2\fboxsep - 2\fboxrule}%

\subsection*{Story 1 (Spotania) - Text}

Queen Porridge of Spotania was born in the year 1012. She came to
power in 1023 and ruled until 1063 when she was overthrown by a revolution.
Following the overthrow she lived in exile until her death in 1092.
Seventeen years into her reign, in 1040, with Chickenalia as an ally,
Spotania attacked Ruritania and started the two year long 'war of
the cream cakes'. Ultimately Spotania and Chickenalia won the war
and the peace treaty was signed in 1042. The war resulted in the acquisition
of valuable mineral resources and associated trade routes and led
to significant improvements in the economy, as noted in many journals
in 1044, and an increase in the population as shown by the 1046 census.
Some important early artistic works included a famous painting of
the 'Battle of the Eclair' by the artist Victoria during the War of
the Cream Cakes in 1041, and the 'Fiona Symphony' composed in 1050.
The invention of Sempahore towers by Daniel in 1053 provided rapid
communication across the whole country. But the census of 1058, showed
continuation of the population growth and there were also many indications
of a serious decline in the economy the same year. Four years later
in 1062 a revolution started. Although initially centred on the capital,
subversives use of the semaphore system led to a rapid spread across
the whole country. The revolution culminated in 1063 with the overthrow
of Queen Porridge and her replacement by a revolutionary council to
rule the whole country. One year later however, in 1064, that council
was quickly defeated, within the space of a year, by a coalition of
Ruritanian rebels and Chickenalia forces in the 'war of the cold vegetables'.
The victors then appointed King Jas to rule from 1065. After living
in exile ex-Queen Porridge of Spotania passed away in 1092. %
\end{minipage}}

\begin{figure}

\includegraphics[scale=0.5]{Table.eps}\caption{Table Test}

\end{figure}


\begin{figure}

\includegraphics[scale=0.5]{Figures/Spotania2}\caption{Story 1 (Spotania) Graphical Form}

\end{figure}

\noindent\fbox{\begin{minipage}[t]{1\columnwidth - 2\fboxsep - 2\fboxrule}%

\subsection*{Story 2 (Brontavia) - Text}

Brontavia was a very prosperous country until the great famine of
1820 and was ruled by King Cedric (born 1790) from his accession in
1810 at the age of 20 until the Hungry Peoples' Revolution in 1821.
Cedric was always a pleasure seeking prince and was ridiculed by his
subjects after being clearly shown misbehaving by the statue \textquotedbl{}Future
King partying\textquotedbl{} created by sculptor Juan Le Lizard in
1808. The great famine resulted from a combination of raging potato
fungus, and purple wheat blight, devastating both of the country's
main food crops. With no harvest to collect, the mainly rural population
not only had no food, but also no work. It was therefore little surprise
that the peasant class revolted and overthrew King Cedric and his
followers, who were mostly city playboys completely indifferent to
the suffering of the peasants. The removal of the king led to a very
long period of anarchy from 1821 to 1825 with rival factions such
as the 'Hungry Zealots', the 'Mighty Marauders' and the Pastie Brigade
battling for overall rule. The Hungry Zealots and the Pastie Brigade
met at the battle of Herbert's Bridge in late 1823. This battle was
commemorated in Van de Vert's famous painting “Bridge of Significance”
(1823). After weeks of fighting, the Zealots triumphed and then marched
to the stronghold of the Mighty Marauders aiming to swiftly consolidate
their victory and take charge overall. The tiring long march for the
Zealots and the exceptional defences prepared by the Marauders however,
put paid to that aim, and the Zealots were effectively wiped out in
the resulting battle which became known, through the songs of the
popular troubadour 'Beardy Jason' as the 'Battle of the Big Bad Bend',
a tune he composed shortly after the battle in 1824, but which became
famous throughout the land during the following year (1825). The Marauders
proved poor rulers, mainly through their lack of organisation and
were not well liked. In 1827 around the middle of their 5 year rule,
an unknown inventor created a device known as the 'Wax Sealed Ballot'
a way of ensuring completely fair and secure elections. This was widely
popularised by the secretive artist, 'Plankrider', who created a symbol
clearly representing the device which was able to be easily copied
in graffiti. The graffiti spread widely over the next year (1828-1829)
appearing almost everywhere and so, recognising the clear desire of
the population, the Marauders arranged a free election in 1829, actually
using the wax sealed ballot. Surprisingly they came a very close second
in the election, only just losing to the People's Party of Brontavia
and the resulting coalition then ruled successfully for the next 20
years and even arranged a memorial ceremony for the death of ex-King
Cedric in 1840 since the population had, by that time, mostly forgiven
him.%
\end{minipage}}

\begin{figure}
\includegraphics[scale=0.27]{Figures/Brontavia-v02}\caption{Story 2 (Brontavia) Graphical Form}
\end{figure}

\begin{table}[h]
\caption{Analysis - Story 2 (Brontavia) text}
\begin{tabular}{cccc>{\centering}p{1.5cm}>{\centering}p{1.5cm}c>{\centering}p{1.5cm}}
\hline
Names & Dates & Lines & Links & \multirow{1}{1.5cm}{Flesch-Kincaid Grade} & Flesch Reading Ease & Words & Distinct Words\tabularnewline
\hline
15 & 12 & 4 & 7 & 17.5 & 33.6 & 471 & 273\tabularnewline
\hline
\end{tabular}
\end{table}


\subsubsection{Questions}

The questions were of three types (numerical (age/date), multiple
choice selection (1 of 2, 3, or 4 choices) , and list ordering) with
12 questions for each story in the Comprehension and Retention elements
and 6 in the initial retention session

Some of the questions were identical between sessions (to enhance
comparison). Tables \ref{tab:Spotania-Questions} and \ref{tab:Brontavia-Questions}
list the questions used for each story and are highlighted in light
gray where two sets have the same question, and darker gray where
all three are identical.

The question types are defined by a combination of the entry in the
column 'Type' which indicates whether the result would be a number
or a selection between 2, 3 or 4 options. Numeric types can be further
subdivided into two digit numbers (question code has third symbol
as 'A') or four digit (question code has third symbol as 'D') and
choice types between simple selection of a single choice (question
code has third symbol as 'C' or 'R') and those questions where the
participant orders three or four options (question code has third
symbol as 'O').

\begin{sidewaystable}
\begin{tabular}{cc>{\centering}p{4cm}cc>{\centering}p{4cm}cc>{\centering}p{4cm}}
\hline
\multicolumn{3}{|c|}{Comprehension} & \multicolumn{3}{c|}{Baseline} & \multicolumn{3}{c|}{Retention}\tabularnewline
\hline
\multicolumn{1}{|c|}{Type} & \multicolumn{1}{c|}{Code} & \multicolumn{1}{>{\centering}p{4cm}|}{Question} & \multicolumn{1}{c|}{Type} & \multicolumn{1}{c|}{Code} & \multicolumn{1}{>{\centering}p{4cm}|}{Question} & \multicolumn{1}{c|}{Type} & \multicolumn{1}{c|}{Code} & \multicolumn{1}{>{\centering}p{4cm}|}{Question}\tabularnewline
\hline
NUMBER & GSA1 & \cellcolor{lightgray}How old in years was Queen Porridge when she
was overthrown? & NUMBER & LLA2 & \cellcolor{lightgray}How old in years was Queen Porridge when she
was overthrown? & NUMBER & LSA12 & \cellcolor{lightgray}How old in years was Queen Porridge when she
was overthrown?\tabularnewline
SELECT2 & GSC2 & Which was the longer war? the 'war of the cold vegetables' or the
'war of the cream cakes'? &  &  &  & SELECT2 & LSC2 & Was Queen Porridge deposed in 1063\tabularnewline
NUMBER & GSA3 & For how many years did Queen Porridge reign? &  &  &  &  &  & \tabularnewline
SELECT3 & GSO4 & \cellcolor{verylightgray}Please put the rulers of Spotania in order
with the most recent at the top & SELECT3 & LLO4 & Please put the following items in order with the most recent at the
top & SELECT3 & LSO11 & \cellcolor{verylightgray}Please put the rulers of Spotania in order
with the most recent at the top\tabularnewline
SELECT4 & GSR5 & \cellcolor{verylightgray}What do you think was the main cause of
the revolution?  &  &  &  & SELECT4 & LSR8 & \cellcolor{verylightgray}What do you think was the main cause of
the revolution? \tabularnewline
SELECT4 & GSR6 & \cellcolor{verylightgray}Which artistic item was created during a
war period? &  &  &  & SELECT4 & LSR9 & \cellcolor{verylightgray}Which artistic item was created during a
war period?\tabularnewline
\cline{3-3} \cline{9-9}
NUMBER & GSD7 & \cellcolor{verylightgray}In what year were Semaphore towers invented? & NUMBER & LLD3 & What year did Queen Porridge come to power? & NUMBER & LSD10 & \cellcolor{verylightgray}In what year were Semaphore towers invented?\tabularnewline
NUMBER & GSA8 & \cellcolor{verylightgray}How old in years was Queen Porridge when
she died? &  &  &  & NUMBER & LSA7 & \cellcolor{verylightgray}How old was Queen Porridge when she died?\tabularnewline
SELECT4 & GSR9 & \cellcolor{verylightgray}Who was ruling in 1065?  &  &  &  & SELECT4 & LSR6 & \cellcolor{verylightgray}Who was ruling in 1065? \tabularnewline
SELECT2 & GSC10 & \cellcolor{lightgray}Was Chickenalia involved in the 1040 attack
on Ruritania? & SELECT2 & LLC5 & \cellcolor{lightgray}Was Chickenalia involved in the 1040 attack
on Ruritania? & SELECT2 & LSC4 & \cellcolor{lightgray}Was Chickenalia involved in the 1040 attack
on Ruritania?\tabularnewline
SELECT4 & GSO11 & \cellcolor{verylightgray}Please place the following items in date
order with the most recent at the top &  &  &  & SELECT4 & LSO3 & \cellcolor{verylightgray}Please place the following items in date
order with the most recent at the top\tabularnewline
SELECT4 & GSR12 & \cellcolor{lightgray}Who painted the 'Battle of the Eclair'?  & SELECT4 & LLR6 & \cellcolor{lightgray}Who painted the 'Battle of the Eclair'? & SELECT4 & LSR5 & \cellcolor{lightgray}Who painted the 'Battle of the Eclair'? \tabularnewline
 &  &  &  &  &  & NUMBER & LSD1 & What year did Queen Porridge come to power?\tabularnewline
 &  &  & SELECT3 & LLT1 & Who was the first ruler named in the Spotania story? &  &  & \tabularnewline
\hline
\end{tabular}

\caption{\label{tab:Spotania-Questions}Story 1 (Spotania) Questions}
\end{sidewaystable}

\begin{sidewaystable}
\begin{tabular}{>{\centering}p{1.9cm}>{\centering}p{1.5cm}>{\centering}p{4cm}>{\centering}p{1.9cm}>{\centering}p{1.5cm}>{\centering}p{4cm}>{\centering}p{1.9cm}>{\centering}p{1.4cm}>{\centering}p{4cm}}
\hline
\multicolumn{3}{|c|}{Comprehension} & \multicolumn{3}{c|}{Baseline} & \multicolumn{3}{c|}{Retention}\tabularnewline
\hline
\multicolumn{1}{|>{\centering}p{1.9cm}|}{Type} & \multicolumn{1}{>{\centering}p{1.5cm}|}{Code} & \multicolumn{1}{>{\centering}p{4cm}|}{Question} & \multicolumn{1}{>{\centering}p{1.9cm}|}{Type} & \multicolumn{1}{>{\centering}p{1.5cm}|}{Code} & \multicolumn{1}{>{\centering}p{4cm}|}{Question} & \multicolumn{1}{>{\centering}p{1.9cm}|}{Type} & \multicolumn{1}{>{\centering}p{1.4cm}|}{Code} & \multicolumn{1}{>{\centering}p{4cm}|}{Question}\tabularnewline
\hline
NUMBER & GBA1 & \cellcolor{verylightgray}How old in years was King Cedric when he
was overthrown by the revolution? & NUMBER & LLA9 & \cellcolor{verylightgray}How old in years was King Cedric when he
was overthrown by the revolution? & NUMBER & LBD10 & In what year was the Wax Sealed Ballot invented?\tabularnewline
SELECT2 & GBC2 & Who did the Hungry Zealots fight first? - The Pastie Brigade or the
Mighty Marauders? & SELECT2 & LLC12 & \cellcolor{verylightgray}Was Purple Wheat blight involved in the
great famine? & SELECT2 & LBC4 & \cellcolor{verylightgray}Was Purple Wheat blight involved in the
great famine?\tabularnewline
NUMBER & GBA3 & For how many years did King Cedric reign? &  &  &  & NUMBER & LBA12 & For how many years did the Mighty Marauders rule?\tabularnewline
SELECT3 & GBO4 & \cellcolor{verylightgray}Please put the rulers of Brontavia in order
with the most recent at the top  &  &  &  & SELECT3 & LBO11 & \cellcolor{verylightgray}Please put the rulers of Brontavia in order
with the most recent at the top \tabularnewline
SELECT4 & GBR5 & \cellcolor{verylightgray}What do you think was the main cause of
the revolution?  &  &  &  & SELECT4 & LBR8 & \cellcolor{verylightgray}What do you think was the main cause of
the revolution? \tabularnewline
SELECT4 & GBR6 & \cellcolor{verylightgray}Which artistic item was created during the
revolution? & SELECT4 & LLR8 & Who was the first ruler named in the Brontavia story? & SELECT4 & LBR9 & \cellcolor{verylightgray}Which artistic item was created during the
revolution?\tabularnewline
NUMBER & GBD7 & In what year was the Wax Sealed Ballot invented? & NUMBER & LLD10 & \cellcolor{verylightgray}What year did King Cedric come to power? & NUMBER & LBD1 & \cellcolor{verylightgray}What year did King Cedric come to power?\tabularnewline
NUMBER & GBA8 & \cellcolor{verylightgray}How old in years was King Cedric when he
died? &  &  &  & NUMBER & LBA7 & \cellcolor{verylightgray}How old was King Cedric when he died?\tabularnewline
SELECT4 & GBR9 & \cellcolor{verylightgray}Who was ruling in 1827?  &  &  &  & SELECT4 & LBR6 & \cellcolor{verylightgray}Who was ruling in 1827? \tabularnewline
SELECT2 & GBC10 & Was King Cedric born in 1800? &  &  &  & SELECT2 & LBC2 & Was King Cedric deposed in 1821\tabularnewline
SELECT4 & GBO11 & \cellcolor{lightgray}Please place the following items in date order
with the most recent at the top & SELECT4 & LLO11 & \cellcolor{lightgray}Please place the following items in date order
with the most recent at the top & SELECT4 & LBO3 & \cellcolor{lightgray}Please place the following items in date order
with the most recent at the top\tabularnewline
SELECT4 & GBR12 & \cellcolor{lightgray}Who painted the 'Bridge of Significance'?  & SELECT4 & LLR13 & \cellcolor{lightgray}Who painted the 'Bridge of Significance'?  & SELECT4 & LBR5 & \cellcolor{lightgray}Who painted the 'Bridge of Significance'? \tabularnewline
\hline
\end{tabular}

\caption{\label{tab:Brontavia-Questions}Story 2 (Brontavia) Questions}
\end{sidewaystable}

The other tests used were either researcher written or obtained from
freely available sources (Creative Commons copyright etc). Table \ref{Sources}
provides details.

\begin{table}
\begin{tabular}{>{\centering}p{0.35\paperwidth}>{\centering}p{0.4\paperwidth}}
\hline
Test & Source\tabularnewline
\hline
Main stories & Researcher written\tabularnewline
Vividness of Imagery - Plymouth Sensory Image Questionnaire - Visual
modality  & \citet{andrade_assessing_2013}\tabularnewline
Visualiser/Verbaliser preferences  & \citet{massa_testing_2006,mayer_three_2003}\tabularnewline
Multimedia preferences & \citet{massa_testing_2006}\tabularnewline
Spatial Rotation Images & \citet{tarr_shepard_metzler.zip_????} for shapes, researcher produced
for rotated letter \tabularnewline
Object Location Memory & Researcher coded - based upon \citet{james_sex_1997}\tabularnewline
\hline
\end{tabular}

\caption{\label{Sources}Test Sources}
\end{table}

Pilot experiments involving the spatial rotation tests showed that
the response time measurement capability built into Limesurvey suffered
from a large variation in measurement accuracy. A controlled trial
of an automated 1000ms response resulted in a mean of 1938ms and s.d.
301. Clearly when it is considered that \citet{shepard_mental_1971}
found mental rotation speeds of around 60 degrees per second while
\citet{ganis_new_2015} quotes 100 degrees per second (albeit for
a slightly different set) a measurement error distribution with an
s.d of 301 will have a significant impact upon the reliability of
measurements that are themselves around 400ms. A javascript program
suitable for embedding into each page of the rotation timing tests
was therefore developed (and incorporated through the scripting process
described earlier). The same controlled trial of this approach resulted
in a measurement mean of 1069ms and s.d. 30.

Javascript incorporated into the relevant page was also used for the
Object location memory test which involved the participant studying
the arrangement of objects (placed in an 8x8 grid) for 10 seconds
and then, after a 2 second blank period, being shown the altered array
and asked to identify the changed objects by mouse click.

\subsection{Procedure}

After a welcome to the first session by researcher and allocation
of a computer, participants were given time to study the background
on a consent form and to sign the consent form. The first web pages
of the study provided background information and the participant was
then asked by the software to enter their demographic information.
The next pages provided an explanation of the experiment and incorporated
practice examples of three of the basic question types The participant
was randomly allocated (by the software) to one of the four sets of
questions (text first x graphics first, starting with Story 1 x starting
with Story 2). The first page in each set provided the story in text
or graphical form and the participant was allowed to take as long
as they liked to study it (they were also told that the information
would be repeated on every subsequent page of the comprehension test).
There was a clear division between the sets of questions for the two
different stories with a similar opportunity to study the text/graphic
on its own provided before the second set of questions started. Self-rating
of mental effort via a 7 point likert scale (1 = extremely low, 7
= extremely high) was included after one example of each question
type in each group (arranged so that the participant was never asked
to rate this on consecutive questions) and an assessment of overall
difficulty using a 5 point scale (1 = extremely easy, 5 = extremely
hard) included after completion of each story . Both scales are in
line with the approach used by \citet{deleeuw_comparison_2008} (to
indicate intrinsic and germane cognitive loads respectively). The
next set of questions used self report to assess Vividness of Visual
Imagery \citep{marks_visual_1973}, Visualiser/Verbaliser and Multimedia
Learning Preference \citep{massa_testing_2006,mayer_three_2003}.
The participant was then presented with pairs of shapes as used by
\citep{shepard_mental_1971} and asked to identify whether these are
the same or different. There were 60 pairs 30 were congruent and 30
incongruent with 3 examples at each rotation angle of 0, 20, 40, 60,
80, 100, 120, 140, 160, 180 degrees. This was followed by 48 pairs
of images of a large letter 'F' again with 24 being identical but
rotated and 24 having one image a rotated mirror of the other (hence
not able to be mentally rotated into alignment). In this case rotation
angles of 40, 80, 120, 160, 200, 240, 280, 320 were used (broadly
in line with \citet{gray_psychopy/psychopy_????}) The participant
also provided a memorable word/letter sequence to form an anonymous
link between the first and second sessions and was given an opportunity
to submit comments about the experiment itself.

The retention tests took place a week later (7+/-2days) - (this period
was chosen to match periods used by \citep{roediger_critical_2011,cepeda_optimizing_2008}
when assessing retention). The second session took a very similar
computer based form to the first session but in there was no display
of the test material. After entering their participant chosen link
from the first session, a set of questions was presented covering
each of the stories in turn (the story that was asked about first
being selected at random). These were followed by similar spatial
ability tests to the first session. Finally two tests of object location
memory were performed where in the first test 6 out of 9 items remained
static and in the second 6 out of 12 remained static.

\section{Results}

\subsection{Participant Accuracy }

. Histograms of the marks (all marks in \% terms) in both conditions
and all three sampling points can be compared in figure \ref{fig:Distributions-of-Marks}.
The means and variances can be compared in figure \ref{fig:Text-and-Graphics}
and the statistics for the resultant distributions can be seen in
table \ref{tab:Basic-statistics-of-results}.

As can be seen the majority (Apart from textual marks in the Retention
session) of the distributions of marks should not be considered as
normal. Non-parametric tests were therefore used in the comparisons.

\begin{table}
\include{Mark_Stats_Table}\caption{\label{tab:Basic-statistics-of-results}Basic-statistics of results
- %
\begin{lyxgreyedout}
Not sure if I need all these but do need to have two line headings
for columns%
\end{lyxgreyedout}
}

\end{table}
\begin{figure}
\includegraphics[scale=0.5]{Figures/marks_barplot}

\caption{\label{fig:Text-and-Graphics}Text and Graphics results for the 3
parts (with error bars)}
\end{figure}

\begin{figure}
\includegraphics[scale=0.5]{Figures/marks_hists}

\caption{\label{fig:Distributions-of-Marks}Distributions of Marks}
\end{figure}

Comparisons between the marks for material presented in the graphical
format (\emph{Mdn} = 83.3) and the text format (\emph{Mdn} = 91.7)
showed no significant difference for comprehension in the first session
(\emph{U = 1576, ns, r = - 0.15}).\emph{ }For the Initial Retention
tests the graphical results (\emph{ Mdn} = 83.3) were again, not significantly
different to those for the text format (\emph{Mdn} = 66.7), (\emph{U
= 1478, ns, r = - 0.0}4)\emph{. }For the retention session one week
later however, on average, the participant remembered slightly more
from information that had been presented in the textual form during
the first session (\emph{Mdn} = 58.3) than from information that had
been presented in Graphical form (\emph{Mdn} = 50.0), (\emph{U = 1651,
p<0.05, r = - 0.}23). %
\begin{lyxgreyedout}
Talk about BASELINE and \citet{hinze_importance_2013}. %
\end{lyxgreyedout}

\begin{table}
\include{All_RPS_RT}

\caption{RPS\_RT include}
\end{table}


\subsection{Response Timings}

\section{Discussion}

Initial Comprehension

Forgetting

Retention

Role of feedback - need in later study

Cognitive load estimates

\subsection{Accuracy/Marks}

The distribution of marks separated into those relating to text material
and those relating to graphical can be seen in Figure \ref{fig:Distributions-of-Marks}

The three plots on the left represent text results, while those on
the right, graphical results. In each group the leftmost plot shows
the comprehension results, the centre, the baseline results, and the
rightmost, the one week retention results. Table \ref{tab:Results-summary--Raw}
summarises these results. It can be seen from the results of the Shapiro-Wilk
test of normality that only the Graphical retention test is above
the 0.05 threshold.

\subsection{Session differences}

Three comparisons (per type - graphical/Text) can also be made between
the initial comprehension session and the baseline retention session,
the baseline and the one week later retention, and the initial comprehension
to the retention session. Of these the most relevant is the baseline
to retention comparison as this is expected to best reflect the level
of learning from the different presentation types. The other two are
included for completeness/discussion only.
\begin{table}
\begin{tabular}{|c|c|c|c|c|}
\hline
Type & Min & Max & Median & Shapiro\tabularnewline
\hline
\hline
T\_Base\_to\_ RPS & -50 & 25 & -17 & .06\tabularnewline
\hline
G\_Base\_to\_ RPS & -67 & 33 & -8.3  & .13\tabularnewline
\hline
T\_Retn\_to\_Base & -67 & 25 & -25 & .18\tabularnewline
\hline
G\_Retn\_to\_Base & -75 & 17 & -17 & .15\tabularnewline
\hline
T\_Retn\_to\_RPS & -75 & -8 & -42 & .05\tabularnewline
\hline
G\_Retn\_to\_RPS & -75 & 33 & -29 & .26\tabularnewline
\hline
\end{tabular}

\caption{Results summary - Mark Differences }
\end{table}

The differences can be seen as boxplots in figure \ref{fig:Boxplot-of-Mark}.
The three plots on the left represent text results, while those on
the right, graphical results. In each group the leftmost plot shows
the difference between the comprehension tests and the baseline tests
(the tests of memory immediately performed during the first session
without the material being re-displayed), the rightmost shows the
difference between the comprehension tests and the retention tests.
The most meaningful plots however show the differences between the
baseline tests and the retention test. \emph{(need the references
here as others have done this) . }These are shown in more detail in
figure \ref{fig:Boxplot-of-Mark-diffs Base to retn}. Since in both
the graphical and text cases the differences in marks between the
Base and Retention scores produced results p >.05 for the Shapiro-Wilks
test, a dependent t-test was performed - On average the participant
remembered slightly more from information that had been presented
in the graphical form during the first session (\emph{Mdn} = -17)
than from information that had been presented in text form (\emph{Mdn}
= -25), \emph{p} = .003, \emph{r = .22}

\begin{figure}
\includegraphics[scale=0.5]{Figures/Mark_differences_boxplot}\caption{\label{fig:Boxplot-of-Mark}Boxplot of Mark Differences}
\end{figure}


\paragraph{Effect of Graphical material on retention }

The marks for the 24 questions in the retention session (12 for each
story) were combined into a total for the case where the version in
the first RPS session was delivered in graphical form and similarly
for the text form. The results were:-

Textual - range: 3:12, M = 8.25, SD = 2.25

Graphical - range: 3:16, M = 9.69, SD = 3.1

For 31 participants the score in the graphical condition was higher,
for 9 the scores were identical and in 12 cases the graphical condition
resulted in a lower retention score.

Calculating the difference between graphical and text scores for each
participant resulted in a distribution with M = 1.44, SD = 2.77.

On average the participant remembered slightly more from information
that had been presented in the graphical form during the first session
(\emph{Mdn} = 7) than from information that had been presented in
text form (\emph{Mdn} = 6), \emph{p} = .003, \emph{r = .29.}

\begin{table}
\begin{tabular}{|c|c|c|c|c|c|c|}
\hline
Name & TNorm & GNorm & DNorm & Paired t\_test & Cohen's d & W\_test\tabularnewline
\hline
\hline
Age & 0.028 & 0.0002 & 0.29 & 0 & 0.72 & 0.0006\tabularnewline
\hline
Choice & 0.0002 & 0.0002 & 0.006 & 0.057 & 0.32 & 0.30\tabularnewline
\hline
Date & 0 & 0 & 0.007 & 0.001 & 0.63 & 0.002\tabularnewline
\hline
Order & 0.001 & 0.004 & 0.32 & 0.97 & 0.28 & 0.08\tabularnewline
\hline
Radio & 0.017 & 0.38 & 0.68 & 0.18 & 0.16 & 0.57\tabularnewline
\hline
Combined & 0.69 & 0.61 & 0.9 & 0 & 0.5 & 0.022\tabularnewline
\hline
\end{tabular}

\caption{Comparison for discussion}
\end{table}

\emph{The tables above suggest that Age and Date questions show significant
differences both with large effect sizes - however I don't understand
why the combined results show similar significance.}

\emph{NOTE: There are 3 age questions, 1 date question, 2 Choice questions,
2 ordering questions, and 4 Radio questions so the others should dominate
2:1}


\appendix

\pagebreak{}

\printbibliography

\end{document}
